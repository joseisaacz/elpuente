\documentclass[10pt,a4paper]{article}
\usepackage[utf8]{inputenc}
\usepackage[latin1]{inputenc}
\usepackage[spanish]{inputenc}
\usepackage{amsmath}
\usepackage{amsfonts}
\usepackage{amssymb}
\usepackage{graphicx}
\usepackage[left=2cm,right=2cm,top=3cm,bottom=2cm]{geometry}
\author{José Isaac Zeledón Jiménez }
\title{Proyecto 0}
\begin{document}
\begin{titlepage}
\begin{center}
\begin{large}
UNIVERSIDAD NACIONAL\\
COSTA RICA \\
\end{large}
\vspace*{1cm}
\begin{large}
Facultad de Ciencias Exactas y Naturales
\end{large} 
\vspace*{1.8cm}\\
Asignatura:\\
\vspace*{2mm}
\begin{large}
Sistemas Operativos\\
\end{large}
\vspace*{12mm}
\begin{large}
\textbf{PROYECTO 0: 
EL PROBLEMA DEL PUENTE ESTRECHO
}\\
\end{large}
\vspace*{1.8cm}
Profesor:\\
\vspace*{5mm}
\begin{large}
Eddy Miguel Ramírez\\
\end{large}
\vspace*{1.8cm}
Estudiantes: \\
\vspace*{5mm}
\begin{large}
José Isaac Zeledón Jiménez\\
Jonathan Estrada Vargas\\
\end{large}
\vspace*{1.8cm}
I CICLO\\
\vspace*{1.8cm}
2019
\end{center}
\end{titlepage}
\section{Introducción}
	Este documento tiene como proposito describir el problema planteado por el profesor, los maneras de abordar este problema y de como resolverlo, los problemas que se afrotaron a la hora de desarrollar la solucion, ademas de las posibles soluciones a los que no se les pudo dar una solucion.\\
De esta manera se puede documentar el proceso de aprendizaje, ademas de dar a concocer las fortalezas y debilidades de los participantes, para asi conocer las areas que el equipo de trabajo debe de reforzar para asi poseer un mejor dominio de los temas que competen a la solucion de este proyecto y sobre todo a el dominio de las habilidades necesarias para la finalizacion del curso de Sistemas Operativos.\\
\section{Descripción del Problema}
El problema que el profesor nos brindo fue el de un problema recurrente en las calles de nuestro pais, lo cual es que en ciertas  carreteras al llegar a un puente, este es del ancho de un automovil lo que causa que solo pueda pasar un auto por sentido. 
Esto genera el problema, de que se deba de encontrar la manera de gestionar el flujo del transito.

Asi que se requirio el desarrollo de una simulacion para representar la situacion del flujo de transito por el puente de una sola via, consistia en representar 3 posibles situaciones que se pueden presentar a la hora de controlar el flujo de los autos. 
\begin{itemize}
\item La tecnica de fuerza bruta.
\item Semaforo
\item Oficial de Transito 
\end{itemize}

\end{document}
\end{document}
